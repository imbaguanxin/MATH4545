\documentclass[10pt]{article}  
\usepackage{graphicx}
\usepackage{geometry}   %设置页边距的宏包

\usepackage{algpseudocode}
\usepackage{amsmath, amssymb, amsthm}
\usepackage{enumerate}
\usepackage{enumitem}
\usepackage{framed}
\usepackage{verbatim}
\usepackage{microtype}
\usepackage{kpfonts}
\usepackage{multicol}
\usepackage{amsfonts}
\newcommand{\overbar}[1]{\mkern 1.5mu\overline{\mkern-1.5mu#1\mkern-1.5mu}\mkern 1.5mu}
\newcommand{\Ib}{\mathbf{I}}
\newcommand{\Pb}{\mathbf{P}}
\newcommand{\Qb}{\mathbf{Q}}
\newcommand{\Rb}{\mathbf{R}}
\newcommand{\Nb}{\mathbf{N}}
\newcommand{\Z}{\mathbf{Z}}
\newcommand{\Zplus}{\mathbf{Z}^+}
\geometry{left=1cm,right=1cm,top=1cm,bottom=1.5cm}  %设置 上、左、下、右 页边距

\begin{document}  
\begin{multicols}{2}
	
\begin{enumerate}
	\item Basic ODEs
	\begin{enumerate}
		\item seperatable: $y'= \frac{F(x)}{G(y)}$\\
		$\Rightarrow \frac{dy}{dx} = \frac{F(x)}{G(y)} \Rightarrow \int G(y)dy = \int F(x)dx$
		
		\item Linear: $y'+p(x)y = q(x)$\\
		Integrating Factor: $e^{\int p(x)dx}$\\
		$\Rightarrow (e^{\int p(x)dx}y)' = e^{\int p(x)dx}q(x)\\ \Rightarrow
		e^{\int p(x)dx}y = \int e^{\int p(x)dx}q(x)dx$
		
		\item $ay'' + by' + cy = 0$ (constant coefficient)\\
		characteristc EQ: $ar^2 + br + c = 0$\\
		$\Delta > 0, y = C_1e^{r_1x} + C_2e^{r_2x}\\
		\Delta = 0, y = C_1e^{rx} + C_2xe^{rx}\\
		\Delta > 0, r = p \pm qi: y = e^{ax}[c_1cos(bx)+c_2sin(bx)]\\$ 
	\end{enumerate}

	\item Fourier Series\\
	Given $F(x), x\in [-L,L]$ write $F(x)$ in a series:
	$$F(x) = \frac{a_0}{2} + \sum_{n = 1}^{\infty}a_n cos(\frac{n\pi x}{L}) + \sum_{n = 1}^{\infty}b_n sin(\frac{n\pi x}{L})$$ where $a_n, b_n$ are constants.
	\begin{enumerate}
		\item  Orthogonality Relations\\
		$\int_{-L}^{L}sin(\frac{n\pi x}{L})cos(\frac{m\pi x}{L}) = 0\\
		\int_{-L}^{L}cos(\frac{n\pi x}{L})cos(\frac{m\pi x}{L}) = 0(m \neq n), L(m = n)\\
		\int_{-L}^{L}sin(\frac{n\pi x}{L})sin(\frac{m\pi x}{L}) = 0(m \neq n), L(m = n)$
		
		\item $a_n = \frac{1}{L}\int_{-L}^{L}F(x)cos(\frac{n\pi x}{L})dx\\
		b_n = \frac{1}{L}\int_{-L}^{L}F(x)sin(\frac{n\pi x}{L})dx, n \in [0, \infty], n \in \Z$
		
		\item Convergece Statement of F.S.\\
		F.S. convergence to the "periodic extension" of F(x) whever F(x) is continous and to the average of $\frac{f(x^+)+f(x^-)}{2}$ at every point.
		
		\item F.S.S and F.C.S of $F(x)$ on $[0,L]$:\\
		F.C.S = $\frac{a_0}{2} + \sum_{n=1}^{\infty}a_ncos\frac{n\pi x}{L},
		a_n = \frac{2}{L}\int_{0}^{L}F(x)cos(\frac{n\pi x}{L})dx$\\
		F.S.S = $\sum_{n=1}^{\infty}b_nsin\frac{n\pi x}{L},
		a_n = \frac{2}{L}\int_{0}^{L}F(x)cos(\frac{n\pi x}{L})dx$\\
		F.C.S $\rightarrow$ even extension, F.S.S $\rightarrow$ odd extension
	\end{enumerate}

	\item Sturm-Liouville Problem
		\begin{enumerate}
			\item Basic Examples of S-L BVP\\
			$f'' + \lambda f = 0, 0\le x \le L$ and Boundary conditions\\
			Def: value $\lambda$ for which the equation with the given boundary ends: has non-trival solution is called eigen value, the corresponding solution is called eigeon functions of the given S-L BVP.\\
			First, general solutions:\\
			$\lambda = 0, f(x) = \alpha x + \beta\\
			\lambda > 0, f(x) = C_1 cos(\sqrt{\lambda}x) + C_2 sin(\sqrt{\lambda}x)\\
			\lambda < 0, f(x) = C_1 cosh(ax) + C_2 sinh(ax), a^2 = -\lambda, a > 0$
			Then, impose the boundary in each case.
			\begin{enumerate}
				\item Boundary COND: $f(0) = 0, f(L) = 0$\\
				e-values: $\lambda_n = \frac{n^2\pi^2}{L^2}, n = 1,2,3$\\
				e-functions: $f_n \sim sin(\frac{n \pi x}{L})$
				\item Boundary COND: $f'(0) = 0, f'(L) = 0$\\
				e-values: $\lambda_n = \frac{n^2\pi^2}{L^2}, n = 0,1,2,3$\\
				e-functions: $f_n \sim cos(\frac{n \pi x}{L})$
				\item Boundary COND: $f(0) = 0, f'(L) = 0$\\
				e-values: $\lambda_n = (\frac{(2n-1)\pi}{2L})^2, n = 1,2,3$\\
				e-functions: $f_n \sim sin(\frac{(2n-1)\pi}{2L}x)$
				\item Boundary COND: $f'(0) = 0, f(L) = 0$\\
				e-values: $\lambda_n = (\frac{(2n-1)\pi}{2L})^2, n = 1,2,3$\\
				e-functions: $f_n \sim cos(\frac{(2n-1)\pi}{2L}x)$
			\end{enumerate}
			\item Regular S-L Problems
			
		\end{enumerate}
	
\end{enumerate}
\newpage
\end{multicols}
\end{document}