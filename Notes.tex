\documentclass[10pt]{article}  
\usepackage{graphicx}
\usepackage{geometry}   %设置页边距的宏包

\usepackage{algpseudocode}
\usepackage{amsmath, amssymb, amsthm}
\usepackage{enumerate}
\usepackage{enumitem}
\usepackage{framed}
\usepackage{verbatim}
\usepackage{microtype}
\usepackage{kpfonts}
\usepackage{multicol}
\usepackage{amsfonts}
\usepackage{array}
\newcommand{\overbar}[1]{\mkern 1.5mu\overline{\mkern-1.5mu#1\mkern-1.5mu}\mkern 1.5mu}
\newcommand{\Ib}{\mathbf{I}}
\newcommand{\Pb}{\mathbf{P}}
\newcommand{\Qb}{\mathbf{Q}}
\newcommand{\Rb}{\mathbf{R}}
\newcommand{\Nb}{\mathbf{N}}
\newcommand{\Fb}{\mathbf{F}}
\newcommand{\Z}{\mathbf{Z}}
\newcommand{\Lap}{\mathcal{L}}
\newcommand{\Zplus}{\mathbf{Z}^+}
\geometry{left=1cm,right=1cm,top=1cm,bottom=1.5cm}  %设置 上、左、下、右 页边距

\begin{document}  
\begin{multicols}{2}
	
\begin{enumerate}
	\item Basic ODEs
	\begin{enumerate}
		\item seperatable: $y'= \frac{F(x)}{G(y)}$\\
		$\Rightarrow \frac{dy}{dx} = \frac{F(x)}{G(y)} \Rightarrow \int G(y)dy = \int F(x)dx$
		
		\item Linear: $y'+p(x)y = q(x)$\\
		Integrating Factor: $e^{\int p(x)dx}$\\
		$\Rightarrow (e^{\int p(x)dx}y)' = e^{\int p(x)dx}q(x)\\ \Rightarrow
		e^{\int p(x)dx}y = \int e^{\int p(x)dx}q(x)dx$
		
		\item $ay'' + by' + cy = 0$ (constant coefficient)\\
		characteristc EQ: $ar^2 + br + c = 0$\\
		$\Delta > 0, y = C_1e^{r_1x} + C_2e^{r_2x}\\
		\Delta = 0, y = C_1e^{rx} + C_2xe^{rx}\\
		\Delta < 0, r = p \pm qi: y = e^{px}[c_1cos(qx)+c_2sin(qx)]$
		
		\item $ay'' + by' + cy = f(x)$ \\
		$y = y_c + y_p$, $y_c$ is solution to homogeneous DIFF EQ\\
		$f(x)$:a polynomial in x or single sin/cos function\\
		$y_p =  x^k$(a polynomial of the same degree), k: $\#$ char eq's zero roots (0,1,2)
		
		$f(x) = e^{ax}$(a polynomial in x)\\
		$y_p =  x^ke^{ax}$(same degree), k: $\#$ char eq's roots $= a$ (0,1,2)
		
		$f(x) = e^{ax}cos(bx)$(poly in x) or $e^{ax}sin(bx)(a poly in x)$\\
		$y_p = x^ke^{ax} [ ($poly in x$)cos(bx) + ($poly in x$)sin(bx) ]$\\
		k: $\#$ char eq's root = $a \pm bi$ (0,1)
	\end{enumerate}

	\item Fourier Series\\
	Given $F(x), x\in [-L,L]$ write $F(x)$ in a series:
	$$F(x) = \frac{a_0}{2} + \sum_{n = 1}^{\infty}a_n cos(\frac{n\pi x}{L}) + \sum_{n = 1}^{\infty}b_n sin(\frac{n\pi x}{L})$$ where $a_n, b_n$ are constants.
	\begin{enumerate}
		\item  Orthogonality Relations\\
		$\int_{-L}^{L}sin(\frac{n\pi x}{L})cos(\frac{m\pi x}{L}) = 0\\
		\int_{-L}^{L}cos(\frac{n\pi x}{L})cos(\frac{m\pi x}{L}) = 0(m \neq n), L(m = n)\\
		\int_{-L}^{L}sin(\frac{n\pi x}{L})sin(\frac{m\pi x}{L}) = 0(m \neq n), L(m = n)$
		
		\item $a_n = \frac{1}{L}\int_{-L}^{L}F(x)cos(\frac{n\pi x}{L})dx\\
		b_n = \frac{1}{L}\int_{-L}^{L}F(x)sin(\frac{n\pi x}{L})dx, n \in [0, \infty], n \in \Z$
		
		\item Convergece Statement of F.S.\\
		F.S. convergence to the "periodic extension" of F(x) whever F(x) is continous and to the average of $\frac{f(x^+)+f(x^-)}{2}$ at every point.
		
		\item F.S.S and F.C.S of $F(x)$ on $[0,L]$:\\
		F.C.S = $\frac{a_0}{2} + \sum_{n=1}^{\infty}a_ncos\frac{n\pi x}{L},
		a_n = \frac{2}{L}\int_{0}^{L}F(x)cos(\frac{n\pi x}{L})dx$\\
		F.S.S = $\sum_{n=1}^{\infty}b_nsin\frac{n\pi x}{L},
		a_n = \frac{2}{L}\int_{0}^{L}F(x)cos(\frac{n\pi x}{L})dx$\\
		F.C.S $\rightarrow$ even extension, F.S.S $\rightarrow$ odd extension
	\end{enumerate}

	\item Sturm-Liouville Problem
		\begin{enumerate}
			\item Basic Examples of S-L BVP\\
			$f'' + \lambda f = 0, 0\le x \le L$ and Boundary conditions\\
			Def: value $\lambda$ for which the equation with the given boundary ends: has non-trival solution is called eigen value, the corresponding solution is called eigeon functions of the given S-L BVP.\\
			First, general solutions:\\
			$\lambda = 0, f(x) = \alpha x + \beta\\
			\lambda > 0, f(x) = C_1 cos(\sqrt{\lambda}x) + C_2 sin(\sqrt{\lambda}x)\\
			\lambda < 0, f(x) = C_1 cosh(ax) + C_2 sinh(ax), a^2 = -\lambda, a > 0$
			Then, impose the boundary in each case.
			\begin{enumerate}
				\item Boundary COND: $f(0) = 0, f(L) = 0$\\
				e-values: $\lambda_n = \frac{n^2\pi^2}{L^2}, n = 1,2,3$\\
				e-functions: $f_n \sim sin(\frac{n \pi x}{L})$
				\item Boundary COND: $f'(0) = 0, f'(L) = 0$\\
				e-values: $\lambda_n = \frac{n^2\pi^2}{L^2}, n = 0,1,2,3$\\
				e-functions: $f_n \sim cos(\frac{n \pi x}{L})$
				\item Boundary COND: $f(0) = 0, f'(L) = 0$\\
				e-values: $\lambda_n = (\frac{(2n-1)\pi}{2L})^2, n = 1,2,3$\\
				e-functions: $f_n \sim sin(\frac{(2n-1)\pi}{2L}x)$
				\item Boundary COND: $f'(0) = 0, f(L) = 0$\\
				e-values: $\lambda_n = (\frac{(2n-1)\pi}{2L})^2, n = 1,2,3$\\
				e-functions: $f_n \sim cos(\frac{(2n-1)\pi}{2L}x)$
			\end{enumerate}
			\item Regular S-L Problems\\
				EQ: $(pf')' + qf + \lambda \sigma f = 0, a < x < b$\\
				Boundary: $k_1f(a) + k_2f'(a) = 0, k_3f(b) + k_4f'(b) = 0$\\
				$$g(x) \sim \sum_{n=1}^{\infty}a_n f_n(x) $$
				$$a_n = \frac{\int_{a}^{b}g(x)f_n(x)\sigma (x)dx}{\int_{a}^{b}\sigma(x) f_n^2(x)dx}$$
			
		\end{enumerate}
	
	\item Method of Separationg: Heat and Wave Equation
		\begin{enumerate}
			\item Heat Equation\\
				$\mu_t = k\mu_{xx}$, $\mu(x,t) = X(x)T(t)$ (usually k = 1)\\
				Seperation$\rightarrow X(x)T'(t) = kX''(x)T(t)$\\
				Let$\frac{T'}{kT} = \frac{kX''}{x} = -\lambda$\\
				We get $X'' + \lambda x = 0$, $T' + k\lambda T = 0$\\
				For second EQ, T $\sim e^{-k\lambda t}$\\
				For first EQ, we apply S-L BVP problem\\
				$\mu(x,t) = \sum_{n = 1}^{\infty}c_n f_n e^{-\lambda_n t}$\\
				Usually, we find the bound. cond. in the 4 fourier series probs.\\
				e.g. bond cond 1:\\
				$\lambda_n = (\frac{n \pi}{L})^2$, $f_n \sim sin(\frac{n \pi x}{L})$\\
				$\mu (x,0) = g(x) = \sum_{n = 1}^{\infty}c_n sin(\frac{n\pi x}{L})$\\
				Fouries formal solution: $\mu (x,t) = \sum_{n = 1}^{\infty}c_nsin(\frac{n \pi x}{L})e^{-\frac{n^2\pi^2}{L^2}t}$ \\
				$\mathbf{note}:$ in bond cond 2, n starts from 0. calculate case 0 seperately.
			\item Wave Equation\\
				$\mu_{tt} = C^2\mu_{xx}, \mu(x,t) = X(x)T(t)$, usually $C = 1$\\
				$\mu (0,t) = 0$, $\mu (L,t)=0$\\
			 	Seperation $\rightarrow XT'' = C^2X''T$\\
				Let $\frac{T''}{c^2T} = \frac{X''}{x} = -\lambda$\\
				We get $X'' + \lambda x = 0$, $T'' =  -\lambda c^2T$\\
				$\mu (x,t) = \sum_{n = 1}^{\infty}(b_{1,n}cos\frac{n\pi ct}{L} + b_{2,n}sin\frac{n\pi c t}{L})sin\frac{n\pi x}{L}$
				\begin{enumerate}
					\item $\mu(x,0) = f(x)$\\
					$f(x) = \sum_{n = 1}^{\infty}b_{1,n}sin\frac{n\pi x}{L}$\\
					$b_{1,n} = \frac{\int_{0}^{L}f(x)sin\frac{n\pi x}{L}dx}{\int_{0}^{L}sin^2\frac{n\pi x}{L}}$\\
					\item $\mu_t(x,0) = g(x)$\\
					$g(x) = \sum_{n = 1}^{\infty} \frac{n\pi c}{L}b_{2,n}sin\frac{n\pi x}{L}$\\
					$\frac{n\pi c}{L}b_{2,n} = \frac{\int_{0}^{L}g(x)sin\frac{n\pi x}{L}dx}{\int_{0}^{L}sin^2\frac{n\pi x}{L}}$
				\end{enumerate}
			\item d'Alembert's solution to the Wave Equation\\
				Given $\mu (x,0) = F(x)$,$\mu_t (x,0) = 0$ we want to get $\mu(x,t)$\\
				Do an odd extention on $F(x)$, then $\mu(x,t) = \frac{F(x+ct) + F(x-ct)}{2}$. Usually c is 1. Simply draw the graph of $F(x+ct)$(shift to left) and $F(x-ct)$(shift to right) and get the average.
	    \end{enumerate}
    
	\item Laplace Equation\\
		Consider the equilibrium temperature in a uniform rectangle(Heat EQ)
		\begin{equation}
			\left\{
			\begin{array}{lr}
			\mu_{xx}(x,y) + \mu_yy(x,y) = 0, \ 0<x<L, \ 0<y<K&\\
			\mu(0,y) = f_1(y), \ \mu(L, y) = f_2(y),  \ 0<y<K &\\
			\mu(x,0)=g_1(x), \ \mu(x,K) = g_2(x), \ 0 < x < L\\
			\end{array}
			\right.
		\end{equation}
		Where we might have first order partial derivatives on those $\mu's$ on 2nd or 3rd line.\\
		Seperate $\rightarrow \mu = XY$, then\\
		$X''Y+XY'' = 0$\\
		$\frac{X''}{X} = -\frac{Y''}{Y}$\\
		we want to create a SL-BVP problem:\\
		Determine $\frac{X''}{X} = -\frac{Y''}{Y} = \lambda$ or $-\lambda$\\
		When $f_1(y) = f_2(y) = 0$ we can form SL-BVP on Y and we choose $-\lambda$. Similarly, when $g_1(x)= g_2(x) = 0$ we can form SL-BVP on X and we choose $\lambda$\\
		$\mathbf{Case1}$: $-\lambda$:($X(0) \ or \ X'(0) = 0, X(L) \ or \ X'(L) = 0$)\\
		\begin{equation}
		\left\{
		\begin{array}{lr}
		X''+\lambda X = 0, \ \ (SL-BVP)&\\
		\mu(x, 0) = f_1(y), \mu(x, K) = f_2(y)\\
		\end{array}
		\right.
		\end{equation}
		Get $\lambda_n$ and $X_n$ from SL-BVP.
		$$\mu(x,y) = \sum_{n = 1}^{\infty} (\alpha_n sinh\frac{n \pi y}{L} + \beta_n cosh\frac{n \pi y}{L}) X_n$$
		For simplicity, we transform to:\\
		$$\mu(x,y) = \sum_{n = 1}^{\infty} (\alpha_n sinh\frac{n \pi y}{L} + \beta_n sinh\frac{n \pi (y-K)}{L}) X_n$$
		Then we get $\mu(x,0)$, $\mu(x,K)$ (as long as they fit the other 2 equations) and try to figure out $\alpha_n$, $\beta_n$ using match or SL-BVP. $\mathbf{Note}:$ in bond cond 2, n starts from 0. calc case 0 seperately.\\
		$\mathbf{Case2}$: $\lambda$:($Y(0) \ or \ Y'(0) = 0, Y(K) \ or \ Y'(K) = 0$)\\
		\begin{equation}
		\left\{
		\begin{array}{lr}
		Y''+\lambda X = 0, \ \ (SL-BVP)&\\
		\mu(0, y) = g_1(y), \mu(L, y) = g_2(x)\\
		\end{array}
		\right.
		\end{equation}
		Get $\lambda_n$ and $Y_n$ from SL-BVP.
		$$\mu(x,y) = \sum_{n = 1}^{\infty} (\alpha_n sinh\frac{n \pi x}{K} + \beta_n cosh\frac{n \pi x}{K}) Y_n$$
		For simplicity, we transform to:\\
		$$\mu(x,y) = \sum_{n = 1}^{\infty} (\alpha_n sinh\frac{n \pi x}{K} + \beta_n cosh\frac{n \pi (x-L)}{K}) Y_n$$
		Then we get $\mu(0, y)$, $\mu(L, y)$ (as long as they fit the other 2 equations) and try to figure out $\alpha_n$, $\beta_n$ using match or SL-BVP. $\mathbf{Note}:$ in bond cond 2, n starts from 0. calc case 0 seperately.
		
	\item Method of Eigenfunction Expansion\\
		Consider\\
		$PDE: \mu_t = k\mu_{xx}(x,t) + q(x,t), 0<x<L, t > 0\\
		BCs: \mu(0,t) = 0, \mu(L,t) = 0, t > 0\\
		IC: \mu(x, 0) = f(x), 0<x<L$
		\begin{enumerate}
			\item BCs $\Rightarrow$ Eigenfunction:  $X_n(x), \lambda_n$\\
			$\Rightarrow \mu(x,t) = \sum_{n = 1}^{\infty}C_n(t)X_n(x)$\\
			Write $q(x,t) = \sum_{n = 1}^{\infty}q_n(t)X_n(x)$
			\item PDE $$\Rightarrow \sum_{n = 1}^{\infty}C'_n(t)X_n(x) = \sum_{n = 1}^{\infty}C_n(t)X''_n(x)+\sum_{n = 1}^{\infty}q_n(t)X_n(x)$$
			$$\Rightarrow \sum_{n = 1}^{\infty}C'_n(t)X_n(x) = -\sum_{n = 1}^{\infty}C_n(t)\lambda_n X_n(x)+\sum_{n = 1}^{\infty}q_n(t)X_n(x)$$
			$$\Rightarrow \sum_{n = 1}^{\infty}[C'_n(t)+C_n(t)\lambda_n]X_n(x) = \sum_{n = 1}^{\infty}q_n(t)X_n(x)$$
			$$\Rightarrow \sum_{n = 1}^{\infty}[C'_n(t)+C_n(t)\lambda_n] = \sum_{n = 1}^{\infty}q_n(t)$$
			Form may vary ($\mu$ is differentiated in other ways)
			\item IC $\Rightarrow \mu(x,0) = f(x) = \sum_{n=1}^{\infty}C_n(0)X_n(x)$ (SL-BVP)\\
			either match up or $C_n(0) = \frac{\int_{0}^{L}f(x)X_n(x)dx}{\int_{0}^{L}X_n^2(x)dx}$
			\item Then we solve functions $C'_n(t)+C_n(t)\lambda_n = q_n(t)$ for each n with initial condition $C_n(0)'s$ and $q_n(0)'s$
		\end{enumerate}
	
	\item Laplace Transform\\
	Def: $\Lap[f(t)] = \int_{0}^{\infty}f(t)e^{-st}dt$\\
	\begin{equation}
	Def: H_a(t) = H(t-a) = 
	\left\{
	\begin{array}{lr}
	1, \ t \ge a&\\
	0, \ t < a\\
	\end{array}
	\right.
	\end{equation}
	Def: $erf(x) = \frac{2}{\pi} \int_{0}^{x} e^{-u^2}du$, $erfc(x) = 1-erf(x)$
	\begin{center}
		\begin{tabular}{|l l|} \hline
		$f(t) = \Lap^{-1}F (t)$		& $F (t) = \Lap f(s)$ \\ \hline
		$f^{(n)}(t)$ (nth derivative)	& $s^n F(s)-s^{n-1}f(0) - \dots - f^{(n-1)}(0)$  \\
		$H(t-a)f(t-a)$	& $e^{-as}F(s)$ \\
		$e^{at}f(t)$	& $F(s-a)$ \\
		$(f*g)(t)$ 		& $F(s)G(s)$ \\
		1 				& $\frac{1}{s} \ (s > 0)$ \\
		$t^n$ (n is positive integer) & $\frac{n!}{s^{n+1}} \ (s>0) $\\
		$e^{at}$ 			& $ \frac{1}{s-a} \ (s > a)$ \\
		$sin(at)$ 		& $\frac{a}{s^2+a^2} \ (s > 0)$ \\
		$cos(at)$		& $\frac{s}{s^2+a^2} \ (s > 0)$ \\
		$sinh(at)$		& $\frac{a}{s^2-a^2} \ (s > |a|)$ \\
		$cosh(at)$		& $\frac{s}{s^2-a^2} \ (s > |a|)$ \\
		$\delta(t-a) \ (a \ge 0)$ & $e^{-as}$ \\
		$t^ne^{at}$ 	& $\frac{n!}{(s-a)^{n+1}}$ \\
		$e^{at}sinbt$ 	& $\frac{b}{((s-a)^2 + b^2)}$ \\
		$e^atcosbt$		& $\frac{s-a}{((s-a)^2 + b^2)}$ \\
		$erfc(\frac{a}{2 \sqrt{t} })$ & $\frac{1}{s} e^{-a \sqrt{s} }$ \\ \hline
		\end{tabular}
	\end{center}
	Example:\\
	$\mu_{tt}(x,t) = \mu_{xx}(x,t) + 1$, $\mu(x,0) = \mu_t(x,0) = -1, \mu(0,t) = t$\\
	$\Lap{\mu_{tt}} = s^2U-s\mu(x,0) - \mu_t(x,0) = s^2U + 1$\\
	$\Lap{\mu_{xx} + 1} = U_{xx} + \frac{1}{s}$\\
	$\Rightarrow U'' - s^2U = 1 - \frac{1}{s}$ (x is variable, s is scalar)\\
	$U_c = k_1e^{sx} + k_2e^{-sx}$, $k_1 = 0$\\
	$U_p = \frac{1 - \frac{1}{s}}{-s^2} = \frac{1}{s^3} - \frac{1}{s^2}$\\
	$\Lap{\mu(0,t)} = \Lap{t} = \frac{1}{s} = U(0,s)$\\
	$\Rightarrow k_2 + \frac{1}{s^3} - \frac{1}{s^2} = \frac{1}{s^2}$\\
	$\Rightarrow U = (\frac{2}{s^2} - \frac{1}{s^3})e^{-sx} - \frac{1}{s^2} + \frac{1}{s^3}$\\
	$\mu = \Lap^{-1}(U) = H(t-x)[2(t-x)-\frac{1}{2}(t-x)^2] - t + \frac{1}{2}t^2$ 
		
		
\end{enumerate}
\newpage
\end{multicols}
\end{document}